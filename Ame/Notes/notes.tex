\documentclass{article}
\usepackage[top=0.75in, bottom=0.75in, left=1.25in, right=1in]{geometry} %formatage%
\usepackage{amsmath} %pour utiliser des maths de base%
\usepackage{amssymb} %pour faire \mathcal{}=>des lettres ''cursives''%
\usepackage{graphicx} %pour importer des images...http://www.tex.ac.uk/cgi-bin/texfaq2html?label=figurehere%
\usepackage{titlesec} %automatique, pour faire des sous-titres moins laids%
\usepackage{cancel}
\usepackage[procnames]{listings}
\usepackage[utf8]{inputenc} 
\usepackage[T1]{fontenc}        %http://tex.stackexchange.com/questions/11897/draw-a-diagonal-arrow-across-an-expression-in-a-formula-to-show-that-it-vanishes%
\usepackage[frenchb]{babel}
\usepackage{xcolor}
\usepackage[squaren]{SIunits}
\usepackage{subcaption} % Avoir plusieurs sous-figures (graphiques) dans une figures et pouvoire les étiqueter
\usepackage{color}
\usepackage{lipsum}
\usepackage{caption}
\usepackage{wasysym}
\usepackage{braket}
\usepackage{mathtools}
\usepackage{bbm}
\usepackage{array}
\usepackage{diagbox}                                 %diagonale dans les tableaux
\usepackage{float}%placer les tableaux et images où tu veux
\usepackage{listings}
\usepackage[utf8]{inputenc}
\usepackage{comment}
\usepackage{pst-node}
\usepackage{enumitem}
\usepackage{graphicx} % Insérer des graphiques
\usepackage{pgfplots}
\pgfplotsset{width=10cm, compat=1.9}
\usetikzlibrary{patterns,decorations.pathreplacing}
\renewcommand{\figurename}{Illustration}
\newcommand{\tikzmark}[2]{%
	\tikz[remember picture,baseline=(#1.base)]
	\node[circle,red,draw,text=black,anchor=center,inner sep=1pt] (#1) {#2};}
\newcommand{\tikzmarkk}[2]{%
	\tikz[remember picture,baseline=(#1.base)]
	\node (#1) {#2};}


\setcounter{secnumdepth}{0} % sections are level 1

\newtheorem{lemme}{Lemme}
\newtheorem{preuve}{Preuve}
\newtheorem{defini}{Définition}
\newtheorem{propo}{Proposition}
\newtheorem{algo}{Algorithme}

\begin{document}
	
		\begin{titlepage}
		\centering % Centre everything on the title page
		
		\scshape % Use small caps for all text on the title page
		
		\vspace*{3\baselineskip} % White space at the top of the page
		
		%------------------------------------------------
		%	Title
		%------------------------------------------------
		
		\rule{\textwidth}{1.6pt}\vspace*{-\baselineskip}\vspace*{2pt} % Thick horizontal rule
		\rule{\textwidth}{0.4pt} % Thin horizontal rule
		
		\vspace{0.75\baselineskip} % Whitespace above the title
		
		{\LARGE Notes de travail \\} % Title
		\vspace{0.75\baselineskip} % Whitespace below the title
		
		\rule{\textwidth}{0.4pt}\vspace*{-\baselineskip}\vspace{3.2pt} % Thin horizontal rule
		\rule{\textwidth}{1.6pt} % Thick horizontal rule
		
		\vspace{2\baselineskip} % Whitespace after the title block
		
		%------------------------------------------------
		%	Subtitle
		%------------------------------------------------
		{\scshape\Large Prof. Étienne Marceau\\} % Editor list
		
		\vspace*{3\baselineskip}
		
		Estimation de copules hiérarchiques pour un modèle de risque collectif \\% Subtitle or further description
		
		\vspace*{3\baselineskip} % Whitespace under the subtitle
		
		%------------------------------------------------
		%	Editor(s)
		%------------------------------------------------
		
		Préparé par
		
		\vspace{0.5\baselineskip} % Whitespace before the editors
		
		{\scshape\Large Alexandre Lepage, 
			Diamilatou N'diaye, Amedeo Zito \\} % Editor list
		
		\vspace*{3\baselineskip}
		
		le 06 juin 2019
		
		\vspace{0.5\baselineskip} % Whitespace below the editor list
		
		\vfill % Whitespace between editor names and publisher logo
		
		%------------------------------------------------
		%	Publisher
		%------------------------------------------------
		
		\includegraphics[height=1.2cm]{UL_P.pdf}\\
		
		Faculté des sciences et de génie\\
		École d'actuariat\\
		Université Laval\\
		Automone 2018       
	\end{titlepage}
	
	\newpage
	\pagestyle{empty}
	
	\tableofcontents
	
	\newpage
	\setcounter{page}{1}
	\pagestyle{plain}
	
	\section{Chaptire 1 - Fonction Maximum de Vraisemblance}
	\subsection{1.1 - Introduction}
	
	On a la copule de clayton $C$:
	$$C(u_0,\dots,u_n;\alpha) = (  u_0^{-\alpha} + \dots + u_n^{-\alpha})^{-\frac{1}{\alpha}}$$
	
	La densité d'une distribution multivariée non absolument continue est donné par:
	\begin{align*}
	f_{N,X_1,\dots,X_n}(n,x_1,\dots,x_n) &= \frac{\partial^n}{\partial x_1 \dots \partial_n} C(F_N(n),F_{X_1}(x_1),\dots,F_{X_n}(x_n))\\
	&   - \frac{\partial^n}{\partial x_1 \dots \partial_n} C(F_N(n-1),F_{X_1}(x_1),\dots,F_{X_n}(x_n))
	\end{align*}
	
	On utilise cette expression dans la fonction de maximum de vraisemblance avec $\Lambda$ comme vecteur de paramètres:
	
	\begin{align*}
	 \mathcal{L}(\Lambda) &= \left( \Pr (N=0)\right)^{n_0} \\
	 & \times \prod_{n_1}^{} f_{N,X_1}(n,x_1;\Lambda)\\
	 & \times \prod_{n_2}^{} f_{N,X_1,X_2}(n,x_1,x_2;\Lambda)\\
	 & \dots\\
	 & \times \prod_{n_k}^{} f_{N,X_1,\dots,X_n}(n,x_1,\dots,x_n;\Lambda)
	\end{align*}
	
	

\end{document}
 