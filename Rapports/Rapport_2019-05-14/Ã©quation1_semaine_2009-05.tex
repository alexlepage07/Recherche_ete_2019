\documentclass[11pt,letterpaper]{report}
\usepackage[utf8]{inputenc}
\usepackage{amsmath}
\usepackage{amsfonts}
\usepackage{amssymb}
\usepackage[left=2.00cm, right=2.00cm]{geometry}
\author{Alexandre Lepage}
\begin{document}
	Pour déterminer les masses de probabilités, on y arrive de la façon suivante :
	\begin{align}
		P(S=0) &= P(N=0) \nonumber \\
		P(S=1) &= P(N=1, X_1=1) = \sum_{k=1}^{3} P(N=1, X_1=1, X_2 = k) \nonumber \\
		P(S=2) &= P(N=1, X_1=2) + P(N=2, X_1=1, X_2=1) \nonumber \\
		P(S=3) &= P(N=1, X_1=3) + P(N=2, X_1=1, X_2=2) + P(N=2, X_1=2, X_2=1) \nonumber \\
		P(S=4) &= P(N=2, X_1=1, X_2=3) + P(N=2, X_1=2, X_2=2) + P(N=2, X_1=3, X_2=1) \nonumber \\
		P(S=5) &= P(N=2, X_1=2, X_2=3) + P(N=2, X_1=3, X_2=2) \nonumber \\
		P(S=6) &= P(N=2, X_1=3, X_2=3) \nonumber 
	\end{align}
	
	De façon générale, l'équation \eqref{prob_conjointe} résume les lignes plus haut.
	
	\begin{equation}
	\gamma(s) = 
		\left\{
			\begin{array}{ll}
				P(N=0) 																				& , s = 0 \\ 
				\sum_{k=1}^{3} P(N=1, X_1=s, X_2=k) + \sum_{k=1}^{3} P(N=2, X_1=k, X_2 = s-k)  		& , s = 1,2,3,4,5,6
			\end{array} 
		\right.\label{prob_conjointe}
	\end{equation} 
	
	Or, si on calcule la probabilité conjointe que l'une des variables prennent des valeurs hors de son domaine, la probabilité donne zéro.\\
	
	Je ne comprend donc pas où est l'erreur...
	
\end{document}